\documentclass{article}
\usepackage{amsmath}
\usepackage{amssymb}
\usepackage{ragged2e}
\usepackage{tabto}
\usepackage[utf8]{inputenc}
\DeclareMathOperator{\Q}{\mathbb{Q}}
\DeclareMathOperator{\R}{\mathbb{R}}
\DeclareMathOperator{\N}{\mathbb{N}}
\DeclareMathOperator{\E}{\mathbb{E}}
\DeclareMathOperator{\Prop}{\mathbb{P}}
\DeclareMathOperator{\1}{1\!\!1}
\DeclareMathOperator{\Bias}{Bias}
\DeclareMathOperator{\MSE}{MSE}
\DeclareMathOperator{\qed}{QED}
\begin{document}
\NumTabs{5}
\section{}
input space:\tab the set of all strings with characters from \(\Sigma\) ( \(\Sigma^*\))\\
feature space:\tab the frequency of substrings of length p occurring (\(\R^{|\Sigma|^p}\))\\
maping:\tab \(\phi^p\)
\begin{flalign*}
K_p(s,t)=\sum_{u\in\Sigma^p}occ(s,u) \cdot occ(t,u) = \left<\phi^p(s),\phi^p(t)\right>
\end{flalign*}
\section{}
dimension feature space: \[|\Sigma|^p\]
 \section{}
 \begin{tabular}{cccc}
	\(\phi^2\)& s  & t &s,t \\
	AA & 0 & 0&0\\
	AT & 2 & 2&4\\
	AG & 0 & 0&0\\
	AC & 0 & 0&0\\
	TA & 1 & 0&0\\
	TT & 0 & 0&0\\
	TG&  2 & 3&6\\
	TC & 0 & 0&0\\
	GA & 0 & 1&0\\
	GT & 0 & 0&0\\
	GG & 0 & 0&0\\
	GC & 2 & 2&4\\
	CA & 1 & 1&1\\
	CT & 1 & 1&1\\
	CG & 0 & 1&0\\
	CC & 0 & 0&0\\
	\\
	\(K_2\)& & & 16
\end{tabular}

\end{document}